\chapter{Background information \label{background}}

\section{Prerequisites \label{prerequisites}}

You need a basic understanding of Python to use \myhdl{}.
If you don't know Python, you will take comfort in knowing
that it is probably one of the easiest programming languages to
learn~\footnote{You must be bored by such claims, but in Python's
case it's true.}. Learning Python is also one of the better time
investments that engineering professionals can make~\footnote{I am not
biased.}.

For beginners, \url{http://www.python.org/doc/current/tut/tut.html} is
probably the best choice for an on-line tutorial. For alternatives,
see \url{http://www.python.org/doc/Newbies.html}.

A working knowledge of a hardware description language such as Verilog
or VHDL is helpful. 

Code examples in this manual are sometimes shortened for clarity.
 Complete executable examples can be found in the distribution directory at 
\file{example/manual/}.

\section{A small tutorial on generators \label{tutorial}}
\index{generators!tutorial on}

Generators are a relatively recent Python feature. They
were introduced in Python~2.2.
Because generators are the key concept in
\myhdl{}, a small tutorial is included a here.

Consider the following nonsensical function:

\begin{verbatim}
def function():
    for i in range(5):
        return i
\end{verbatim}

You can see why it doesn't make a lot of sense. As soon as the first
loop iteration is entered, the function returns:

\begin{verbatim}
>>> function()
0
\end{verbatim}

Returning is fatal for the function call. Further loop iterations
never get a chance, and nothing is left over from the function call
when it returns.

To change the function into a generator function, we replace
\keyword{return} with \keyword{yield}:

\begin{verbatim}
def generator():
    for i in range(5):
        yield i
\end{verbatim}

Now we get:

\begin{verbatim}
>>> generator()
<generator object at 0x815d5a8>
\end{verbatim}

When a generator function is called, it returns a generator object. A
generator object supports the iterator protocol, which is an expensive
way of saying that you can let it generate subsequent values by
calling its \function{next()} method:

\begin{verbatim}
>>> g = generator()
>>> g.next()
0
>>> g.next()
1
>>> g.next()
2
>>> g.next()
3
>>> g.next()
4
>>> g.next()
Traceback (most recent call last):
  File "<stdin>", line 1, in ?
StopIteration
\end{verbatim}

Now we can generate the subsequent values from the for loop on demand,
until they are exhausted. What happens is that the
\keyword{yield} statement is like a
\keyword{return}, except that it is non-fatal: the generator remembers
its state and the point in the code when it yielded. A higher order
agent can decide when to get a further value by calling the
generator's \function{next()} method. We can say that generators are
\dfn{resumable functions}.

If you are familiar with hardware description languages, this may ring
a bell. In hardware simulations, there is also a higher order agent,
the Simulator, that interacts with such resumable functions; they are
called 
\index{VHDL!process}%
\dfn{processes} in VHDL and 
\index{Verilog!always block}%
\dfn{always blocks} in
Verilog. Like in those languages, Python generators provide an elegant
and efficient method to model concurrency, without having to resort to
some form of threading.

The use of generators to model concurrency is the first key concept in
\myhdl{}. The second key concept is a related one: in \myhdl{}, the
yielded values are used to specify the conditions on which the
generator should wait before resuming. In other words, \keyword{yield}
statements work as generalized 
    \index{sensitivity list}%
sensitivity lists. 

For more info about generators, consult the on-line Python
documentation, e.g. at \url{http://www.python.org/doc/2.2.2/whatsnew}. 


\section{About decorators \label{deco}}
\index{decorators!about}

Python 2.4 introduced a new feature called decorators. MyHDL 0.5 defines
a number of decorators to facilitate hardware descriptions, but many users may
not be familiar with the concept. Therefore, an introduction
is included here.

A decorator consists of special syntax in front of a function
declaration. It refers to a decorator function. The decorator
function automatically transforms the declared function into some
other callable object.

A decorator function \function{deco} is used in a decorator statement as follows:

\begin{verbatim}
@deco
def func(arg1, arg2, ...):
    <body>

This code is equivalent to the following template:

def func(arg1, arg2, ...):
    <body>
func = deco(func)
\end{verbatim}

Note that the decorator statement goes directly in front of the
function declaration, and that the function name func is automatically
reused.

MyHDL 0.5 uses decorators to create ready-to-simulate generators
from local (embedded) function definitions. Their functionality
and usage will be described extensively in this manual.

For more info about Python decorators, consult the on-line Python
documentation, e.g. at \url{http://www.python.org/doc/2.4/whatsnew/node6.html}.

\begin{notice}[warning]
Because MyHDL 0.5 uses decorators, it requires Python 2.4 or a
later version.
\end{notice}
