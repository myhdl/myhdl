\documentclass{manual}
\usepackage{palatino}
\renewcommand{\ttdefault}{cmtt}
\renewcommand{\sfdefault}{cmss}
\newcommand{\myhdl}{{MyHDL}}

\title{The \myhdl\ manual}

\author{Jan Decaluwe}
\authoraddress{
	\strong{Jan Decaluwe}\\
	Email: \email{jan@jandecaluwe.com}
}

\date{February 14, 2003}	% XXX update before release!
\release{0.1}			% software release, not documentation
\setreleaseinfo{}		% empty for final release
\setshortversion{0.1}		% major.minor only for software


\begin{document}

\maketitle

Copyright \copyright{} 2003-2006 Jan Decaluwe.
All rights reserved.


\begin{abstract}

\noindent

\myhdl\ is a Python package for using Python as a hardware description
language. Popular hardware description languages, like Verilog and
VHDL, are compiled languages. \myhdl\ with Python could be viewed as a
"scripting language" counterpart of such languages. However, Python is
more accurately described as a very high level language
(VHLL). \myhdl\ users have access to the amazing power and elegance of
Python in their modeling work.

The key idea behind \myhdl\ is to use Python generators to model the
concurrency required in hardware descriptions. As generators are a
recent Python feature, \myhdl\ requires Python 2.2.2. or higher.

\myhdl\ 0.1 is the initial public release of the package. It can be
used experiment with high level modeling, and with verification
techniques such as unit testing. 

In a future release, \myhdl\ will
hopefully be coupled to hardware simulators for languages such as
Verilog and VHDL. That would turn \myhdl\ into a powerful hardware
verification language.



\end{abstract}

\tableofcontents

\chapter{Background information \label{background}}

\section{Prerequisites \label{prerequisites}}

You need a basic understanding of Python to use \myhdl{}.
If you don't know Python, you will take comfort in knowing
that it is probably one of the easiest programming languages to
learn~\footnote{You must be bored by such claims, but in Python's
case it's true.}. Learning Python is also one of the better time
investments that engineering professionals can make~\footnote{I am not
biased.}.

For beginners, \url{http://www.python.org/doc/current/tut/tut.html} is
probably the best choice for an on-line tutorial. For alternatives,
see \url{http://www.python.org/doc/Newbies.html}.

A working knowledge of a hardware description language such as Verilog
or VHDL is helpful. 

\section{A small tutorial on generators \label{tutorial}}
\index{generators!tutorial on|(}

Generators are a recent Python feature, introduced in
Python 2.2. Therefore, there isn't a lot of tutorial material
available yet. Because generators are the key concept in
\myhdl{}, I include a small tutorial here.

Consider the following nonsensical function:

\begin{verbatim}
def function():
    for i in range(5):
        return i
\end{verbatim}

You can see why it doesn't make a lot of sense. As soon as the first
loop iteration is entered, the function returns:

\begin{verbatim}
>>> function()
0
\end{verbatim}

Returning is fatal for the function call. Further loop iterations
never get a chance, and nothing is left over from the function call
when it returns.

To change the function into a generator function, we replace
\keyword{return} with \keyword{yield}:

\begin{verbatim}
def generator():
    for i in range(5):
        yield i
\end{verbatim}

Now we get:

\begin{verbatim}
>>> generator()
<generator object at 0x815d5a8>
\end{verbatim}

When a generator function is called, it returns a generator object. A
generator object supports the iterator protocol, which is an expensive
way of saying that you can let it generate subsequent values by
calling its \function{next()} method:

\begin{verbatim}
>>> g = generator()
>>> g.next()
0
>>> g.next()
1
>>> g.next()
2
>>> g.next()
3
>>> g.next()
4
>>> g.next()
Traceback (most recent call last):
  File "<stdin>", line 1, in ?
StopIteration
\end{verbatim}

Now we can generate the subsequent values from the for loop on demand,
until they are exhausted. What happens is that the
\keyword{yield} statement is like a
\keyword{return}, except that it is non-fatal: the generator remembers
its state and the point in the code when it yielded. A higher order
agent can decide when to get a further value by calling the
generator's \function{next()} method. We can say that generators are
\dfn{resumable functions}.

If you are familiar with hardware description languages, this may ring
a bell. In hardware simulations, there is also a higher order agent,
the Simulator, that interacts with such resumable functions; they are
called \dfn{processes} in VHDL and \dfn{always blocks} in
Verilog. Like in those languages, Python generators provide an elegant
and efficient method to model concurrency, without having to resort to
some form of threading.

The use of generators to model concurrency is the first key concept in
\myhdl{}. The second key concept is a related one: in \myhdl{}, the
yielded values are used to specify the conditions on which the
generator should wait before resuming. In other words, \keyword{yield}
statements work as generalized 
    \index{sensitivity list}%
sensitivity lists. 

If you want to know more about generators, consult the on-line Python
documentation, e.g. at \url{http://www.python.org/doc/2.2.2/whatsnew}. 

\begin{notice}[warning]
As mentioned earlier, generators were introduced in Python 2.2. In
that version, they were introduced as a ``future'' feature that has to
be enabled explicitly. In Python 2.3, which is the latest stable
Python version at the time of this writing, generators are enabled by
default.

Besides generators, Python 2.3 has several other interesting new
features, and it runs 25--35\% faster than Python 2.2. For this
reasons, I recommend MyHDL users to upgrade to Python 2.3. The next
major release of MyHDL will rely on on the new features and will
require Python 2.3 or higher.

In the mean time, if you use Python 2.2, you have to include the
following line at the start of all code that defines generator
functions:

\begin{verbatim}
from __future__ import generators
\end{verbatim}

From Python 2.3 on, this line may still be in the code, though it will
not have an effect anymore.
\end{notice}
\index{generators!tutorial on|)}


\chapter{Introduction to \myhdl\ }

\section{A basic \myhdl\ simulation}

We will introduce \myhdl\ with a classical \code{Hello World} style
example. Here are the contents of a \myhdl\ simulation script called
\file{hello1.py}:

\begin{verbatim}
from myhdl import delay, now, Simulation

def sayHello():
    while 1:
        yield delay(10)
        print "%s Hello World!" % now()

gen = sayHello()
sim = Simulation(gen)
sim.run(30)

\end{verbatim}

When we run this simulation, we get the following output: 

\begin{verbatim}
% python hello1.py
10 Hello World!
20 Hello World!
30 Hello World!
StopSimulation: Simulated for duration 30

\end{verbatim}

The first line of the script imports a
number of objects from the \code{myhdl} package. In good Python style, and
unlike most other languages, we can only use identifiers that are
\emph{literally} defined in the source file \footnote{I don't want to
explain the \samp{import *} syntax}. 

Next, we define a generator function called
\code{sayHello}. This is a generator function (as opposed to
a classic Python function) because it contains a \keyword{yield}
statement (instead of \keyword{return} statement). In \myhdl\, a
\keyword{yield} statement has a similar purpose as a \keyword{wait}
statement in VHDL: the statement suspends execution of the function,
and its clauses specify when the function should resume. In this case,
there is a \code{delay} clause, that specifies the required delay.

To make sure that the generator runs ``forever'', we wrap its behavior
in a \code{while 1} loop. This is a standard Python idiom, and it is
the \myhdl\ equivalent of a Verilog \keyword{always} block or a
VHDL \keyword{process}.

In \myhdl\, the basic simulation objects are generators. Generators
are the returned from calling generator functions. For example, variable
\code{gen} refers to a generator. To simulate this generator, we pass
it as an argument to a \class{Simulation} object constructor.  We then
run the simulation for the desired amount of time.


\section{Concurrent generators and signals}

In the previous section, we simulated a single generator. Of course,
real hardware descriptions are not like that: in fact, they are
typically massively concurrent. \myhdl\ supports this by allowing an
arbitrary number of concurrent generators. More specifically, a
\class{Simulation} constructor can take an arbitrary number of
arguments, each of which can be a generator or a nested list of
generators.

With concurrency comes the problem of deterministic
communication. Therefore, hardware languages use special objects to
support deterministic communication between concurrent
regions. \myhdl\ has as \class{Signal} object which is roughly
modeled after VHDL signals.

We will demonstrate these concepts by extending and modifying our
first example. We introduce a clock signal, driven by a second
generator: 

\begin{verbatim}
clk = Signal(0)

def clkGen():
    while 1:
        yield delay(10)
        clk.next = 1
        yield delay(10)
        clk.next = 0

\end{verbatim}

The \code{clk} signal is constructed with an initial value
\code{0}. In the clock generator function \code{clkGen}, it is
continuously assigned a new value after a certain delay. In \myhdl{},
the new value of a signal is specified by assigning to its
\code{next} attribute. This is the \myhdl\ equivalent of VHDL signal
assignments and Verilog's non-blocking assignments.

The \code{sayHello} generator function is modified to wait for a
rising edge of the clock instead of a delay:

\begin{verbatim}
def sayHello():
    while 1:
        yield posedge(clk)
        print "%s Hello World!" % now()

\end{verbatim}

Waiting for a clock edge is achieved with a second form of a
\keyword{yield} statement: \samp{yield posedge(\var{signal})}. 
A \class{Simulation} will suspend the generator as that point, and
resume it when there is a rising edge on the signal.

The \class{Simulation} is now constructed with 2 generator arguments:

\begin{verbatim}
sim = Simulation(clkGen(), sayHello())
sim.run(50)

\end{verbatim}

When we run this simulation, we get:

\begin{verbatim}
% python hello2.py
10 Hello World!
30 Hello World!
50 Hello World!
StopSimulation: Simulated for duration 50

\end{verbatim}


\section{Parameters, instantiations and hierarchy}

So far, the generator function examples had no parameters. For
example, the \code{clk} signal was defined in the enclosing scope of
the generator functions. However, to make the code reusable we will
want to pass arguments through a parameter list. For example, we can
change the clock generator function to make it more general
and reusable, as follows:

\begin{verbatim}
def clkGen(clock, period=20):
    lowTime = int(period/2)
    highTime = period - lowTime
    while 1:
        yield delay(lowTime)
        clock.next = 1
        yield delay(highTime)
        clock.next = 0

\end{verbatim}

The clock signal is now a parameter of the function. Also, the clock
period is a parameter with a default value of \code{20}. This
\var{period} is an optional parameter; if it is not specified in a
call, the default value will be used.

Similarly, the \code{sayHello} function can be made more general:

\begin{verbatim}
def sayHello(clock, to="World!"):
    while 1:
        yield posedge(clock)
        print "%s Hello %s" % (now(), to)

\end{verbatim}

We can create any number of generators by calling generator functions
with the appropriate parameters. This is very similar to the concept of
\dfn{instantiation} in hardware description languages and we will use
the same terminology in \myhdl{}. Hierarchy can be modeled by defining
the instances in a higher-level function, and returning them. For
example: 

\begin{verbatim}
def talk():

    clk1 = Signal(0)
    clk2 = Signal(0)
    
    clkGen1 = clkGen(clk1)
    clkGen2 = clkGen(clock=clk2, period=19)
    sayHello1 = sayHello(clock=clk1)
    sayHello2 = sayHello(to="MyHDL", clock=clk2)

    return clkGen1, clkGen2, sayHello1, sayHello2

\end{verbatim}
Like in standard Python, positional or named parameter association can
be used in instantiations, or a mix of the two \footnote{All
positional parameters have to come before any named parameter.}. All
these styles are demonstrated in the example above. Like in hardware
description languages, named association can be very useful if there
are a lot of parameters, as the parameter order does not matter in
that case.

\class{Simulation} constructor arguments can also be sequences of
generators. In this way, they support hierarchy: the return value of a
higher-level instantiating function can directly be used an
argument. For example:

\begin{verbatim}
sim = Simulation(talk())
sim.run(50)

\end{verbatim}

This produces the following output:

\begin{verbatim}
% python greetings.py
9 Hello MyHDL
10 Hello World!
28 Hello MyHDL
30 Hello World!
47 Hello MyHDL
50 Hello World!
StopSimulation: Simulated for duration 50

\end{verbatim}


\section{Bit-oriented operations}

Hardware design involves dealing with bits and bit-oriented
operations. The standard Python \class{int} has most of the desired
features, but lacks support for indexing and slicing. Therefore,
\myhdl\ provides a \class{intbv} class. It works
transparently as an integer and with integers, and like \class{int},
offers access to the underlying a 2's complement representation for
bitwise operations. In addition, it is a mutable type that provides
indexing and slicing operations, and some additional bit-oriented
support such as concatenation.

As an example, we will consider the design of a Gray encoder. The
following code is a Gray encoder modeled in \myhdl{}:

\begin{verbatim}
def bin2gray(width, B, G):
    """ Gray encoder.

    width -- bit width
    B -- input intbv signal, binary encoded
    G -- output intbv signal, Gray encoded
    """
    while 1:
        yield B
        for i in range(width):
            G.next[i] = B.val[i+1] ^ B.val[i]

\end{verbatim}

This code introduces a few new concepts. The string in triple quotes
at the start of the function is a \dfn{doc string}. This is standard
Python practice for the structured documentation of code. Moreover, we
use a third form of the \keyword{yield} statement:
\samp{yield \var{signal}}. This specifies that the generator should
resume whenever \var{signal} changes value. This is typically used to
describe combinatorial logic.

The actual code contains bit indexing operations and an exclusive-or
operator as required for a Gray encoder. Moreover, the code shows how
the actual value of a signal is accessed through the signal's
\var{val} attribute. In \myhdl{}, unlike traditional hardware
description languages, signals are explicitly modeled as composite
objects whose current and next values are accessed through attributes.

\chapter{Modeling techniques}

\section{RTL modeling}
The present section describes how \myhdl\ supports RTL style modeling
as is typically used for synthesizable models in Verilog or VHDL. This
section is included mainly for illustrative purposes, as this modeling
style is well known and understood.

\subsection{Combinatorial logic}

\subsubsection{Template}

Combinatorial logic is described with a generator function code template as
follows: 

\begin{verbatim}
def combinatorialLogic(<arguments>)
    while 1:
        yield <input signal arguments>
        <functional code>

\end{verbatim}

The overall code is wrapped in a \code{while 1} statement to keep the
generator alive. All input signals are clauses in the \code{yield}
statement, so that the generator resumes whenever one of the inputs
changes. 

\subsubsection{Example}

The following is an example of a combinatorial multiplexer:

\begin{verbatim}
def mux(z, a, b, sel):
    """ Multiplexer.
    
    z -- mux output
    a, b -- data inputs
    sel -- control input: select a if asserted, otherwise b
    """
    while 1:
        yield a, b, sel
        if sel == 1:
            z.next = a
        else:
            z.next = b

\end{verbatim}

To verify, let's simulate this logic with some random patterns. The
\code{random} module in Python's standard library comes in handy for
such purposes. The function \code{randrange(\var{n})} returns a random
natural integer smaller than \var{n}. It is used in the test bench
code to produce random input values:

\begin{verbatim}
from random import randrange

(z, a, b, sel) = [Signal(0) for i in range(4)]

MUX_1 = mux(z, a, b, sel)

def test():
    print "z a b sel"
    for i in range(8):
        a.next, b.next, sel.next = randrange(8), randrange(8), randrange(2)
        yield delay(10)
        print "%s %s %s %s" % (z, a, b, sel)
        
Simulation(MUX_1, test()).run() 
   
\end{verbatim}

Because of the randomness, the simulation output varies between runs
\footnote{It also possible to have a reproducible random output, by
explicitly providing a seed value. See the documentation of the
\code{random} module}. One particular run produced the following
output:

\begin{verbatim}
% python mux.py
z a b sel
6 6 1 1
7 7 1 1
7 3 7 0
1 2 1 0
7 7 5 1
4 7 4 0
4 0 4 0
3 3 5 1
StopSimulation: No more events
\end{verbatim}


\subsection{Sequential logic}

\subsubsection{Template}
Sequential RTL models are sensitive to a clock edge. In addition, they
may be sensitive to a reset signal. We will describe one of the most
common patterns: a template with a rising clock edge and an
asynchronous reset signal. Other templates are similar.

\begin{verbatim}
def sequentialLogic(<arguments>, clock, ..., reset, ...)
    while 1:
        yield posedge(clock), negedge(reset)
        if reset == <active level>:
            <reset code>
        else:
            <functional code>

\end{verbatim}


\subsubsection{Example}
The following code is a description of an incrementer with enable, and
an asynchronous power-up reset.

\begin{verbatim}
ACTIVE_LOW, INACTIVE_HIGH = 0, 1

def Inc(count, enable, clock, reset, n):
    """ Incrementer with enable.
    
    count -- output
    enable -- control input, increment when 1
    clock -- clock input
    reset -- asynchronous reset input
    n -- counter max value
    """
    while 1:
        yield posedge(clock), negedge(reset)
        if reset == ACTIVE_LOW:
            count.next = 0
        else:
            if enable:
                count.next = (count + 1) % n

\end{verbatim}

For the test bench, we will use an independent clock generator, stimulus
generator, and monitor. After applying enough stimulus patterns, we
can raise the \code{myhdl.StopSimulation} exception to stop the
simulation run. The test bench for a small incrementer and a small
number of patterns is a follows:

\begin{verbatim}
count, enable, clock, reset = [Signal(intbv(0)) for i in range(4)]

INC_1 = Inc(count, enable, clock, reset, n=4)

def clockGen():
    while 1:
        yield delay(10)
        clock.next = not clock

def stimulus():
    reset.next = ACTIVE_LOW
    yield negedge(clock)
    reset.next = INACTIVE_HIGH
    for i in range(12):
        enable.next = min(1, randrange(3))
        yield negedge(clock)
    raise StopSimulation

def monitor():
    print "enable  count"
    yield posedge(reset)
    while 1:
        yield posedge(clock)
        yield delay(1)
        print "   %s      %s" % (enable, count)
        
Simulation(clockGen(), stimulus(), monitor(), INC_1).run()

\end{verbatim}

The simulation produces the following output:
\begin{verbatim}
% python inc.py
enable  count
   0      0
   1      1
   0      1
   1      2
   1      3
   1      0
   0      0
   1      1
   0      1
   0      1
   0      1
   1      2
StopSimulation
\end{verbatim}

\section{High level modeling}

\begin{quote}
\em
This section on high level modeling should become an exciting part of
the manual, but it doesn't contain a lot of material just yet. One
reason is that I concentrated on the groundwork first. Moreover,
though I expect Python to offer some very powerful capabilities in
this domain, I'm just starting to experiment and learn myself. For
example, note that so far we haven't used classes (nor meta-classes
:-)) yet, even though Python has a very powerful object-oriented
model.
\end{quote}

\subsection{Modeling memories with built-in types}

Python has powerful built-in data types that can be useful to model
hardware memories. This can be merely a matter of putting an interface
around some data type operations.

For example, a \dfn{dictionary} comes in handy to model sparse memory
structures. (In other languages, this data type is called \dfn{
associative array}, or \dfn{hash table}.) A sparse memory is one in
which only a small part of the addresses is used in a particular
application or simulation. Instead of statically allocating the full
address space, which can be large, it is better to dynamically
allocate the needed storage space. This is exactly what a dictionary
provides. The following is an example of a sparse memory model:

\begin{verbatim}


def sparseMemory(dout, din, addr, we, en, clk):

    """ Sparse memory model based on a dictionary.

    Ports:
    dout -- data out
    din -- data in
    addr -- address bus
    we -- write enable: write if 1, read otherwise
    en -- interface enable: enabled if 1
    clk -- clock input
    
    """
    memory = {}
    while 1:
        yield posedge(clk)
        if not en:
            continue
        if we:
            memory[addr] = din.val
        else:
            dout.next = memory[addr]

\end{verbatim} 

Note how we use the \code{val} attribute of the \code{din} signal, as
we don't want to store the signal object itself, but its current
value. (In many cases, \myhdl\ can use a signal's value automatically
when there is no ambiguity: for example, this happens whenever a
signal is used in expressions. When in doubt, you can always use the
\code{val} attribute explicitly.)

As a second example, we will demonstrate how to use a list to model a
synchronous fifo:

\begin{verbatim}
def fifo(dout, din, re, we, empty, full, clk, maxFilling=sys.maxint):

    """ Synchronous fifo model based on a list.

    Ports:
    dout -- data out
    din -- data in
    re -- read enable
    we -- write enable
    empty -- empty indication flag
    full -- full indication flag
    clk -- clock input

    Optional parameter:
    maxFilling -- maximum fifo filling, "infinite" by default

    """
    memory = []
    while 1:
        yield posedge(clk)
        if we:
            memory.insert(0, din.val)
        if re:
            dout.next = memory.pop()
        empty.next = (len(memory) == 0)
        full.next = (len(memory) == maxFilling)

\end{verbatim}

Again, the model is merely a \myhdl\ interface around some operations
on a list: \function{insert()} to insert entries, \function{pop()} to
retrieve them, and \function{len()} to get the size of a Python
object.

\subsection{Modeling errors using exceptions}

In the previous section, we used Python data types for modeling. If
such a type is used inappropriately, Python's run time error system
will come into play. For example, if we access an address in the
\function{sparseMemory} model that was not initialized before, we will
get a traceback similar to the following (some lines omitted for
clarity):

\begin{verbatim}
Traceback (most recent call last):
...
  File "sparseMemory.py", line 30, in sparseMemory
    dout.next = memory[addr]
KeyError: 51

\end{verbatim}

Similarly, if the \code{fifo} is empty, and we attempt to read from
it, we get:

\begin{verbatim}
Traceback (most recent call last):
...
  File "fifo.py", line 34, in fifo
    dout.next = memory.pop()
IndexError: pop from empty list

\end{verbatim}

Instead of these low level errors, it may be preferable to define
errors at the functional level. In Python, this is typically done by
defining a custom \code{Error} exception at the module level, by
subclassing the standard \code{Exception} class. This exception is
then raised explicitly when an error condition occurs.

For example, we can change function \function{sparseMemory} as follows
(with the doc string is omitted for brevity):

\begin{verbatim}
class Error(Exception):
    pass

def sparseMemory(dout, din, addr, we, en, clk):
    memory = {}
    while 1:
        yield posedge(clk)
        if not en:
            continue
        if we:
            memory[addr] = din.val
        else:
            try:
                dout.next = memory[addr]
            except KeyError:
                raise Error, "Uninitialized address %s" % hex(addr)

\end{verbatim}

This works by catching the low level data type exception, and raising
the custom exception with an appropriate error message instead.  If
the \function{sparseMemory} function is defined in a module with the
same name, an access error is now reported as follows:

\begin{verbatim}
Traceback (most recent call last):
...
  File "sparseMemory.py", line 57, in sparseMemory
    raise Error, "Initialize address %s" % hex(addr)
sparseMemory.Error: Initializes address 0x33

\end{verbatim}

Likewise, the \function{fifo} function can be adapted as follows, to
report underflow and overflow errors:

\begin{verbatim}
class Error(Exception):
    pass

def fifo(dout, din, re, we, empty, full, clk, maxFilling=sys.maxint):
    memory = []
    while 1:
        yield posedge(clk)
        if we:
            memory.insert(0, din.val)
        if re:
            try:
                dout.next = memory.pop()
            except IndexError:
                raise Error, "Underflow -- Read from empty fifo"
        empty.next = (len(memory) == 0)
        full.next = (len(memory) == maxFilling)
        if len(memory) > maxFilling:
            raise Error, "Overflow -- Max filling %s exceeded" % maxFilling

\end{verbatim}

In this case, the underflow error is detected as before, by catching a
low level exception on the list data type. On the other hand, the
overflow error is detected by a regular check on the length of the
list.


\begin{quote}
\em
A lot to be added ...
\end{quote}

\chapter{Unit testing}

\section{Introduction}

Many aspects in the design flow of modern digital hardware design can
be viewed as a special kind of software development. From that
viewpoint, it is a natural question whether advances in software
design techniques can not also be applied to hardware design.

One software design approach that gets a lot of attention recently is
\emph{Extreme Programming} (XP). It is a fascinating set of techniques and
guidelines that often seems to go against the conventional wisdom. On
other occasions, XP just seems to emphasize the common sense, which
doesn't always coincide with common practice. For example, XP stresses
the importance of normal workweeks, if we are to have the
fresh mind needed for good software development.

It is not my intention nor qualification to present a tutorial on
Extreme Programming. Instead, in this section I will highlight one XP
concept which I think is very relevant to hardware design: the
importance and methodology of unit testing.

\section{The importance of unit tests}

Unit testing is one of the corner stones of Extreme Programming. Other
XP concepts, such as collective ownership of code and continuous
refinement, are only possible by having unit tests. Moreover, XP
emphasizes that writing unit tests should be automated, that they should
test everything in every class, and that they should run perfectly all
the time. 

I believe that these concepts apply directly to hardware design. In
addition, unit tests are a way to manage simulation time. For example,
a state machine that runs very slowly on infrequent events may be
impossible to verify at the system level, even on the fastest
simulator. On the other hand, it may be easy to verify it exhaustively
in a unit test, even on the slowest simulator.

It is clear that unit tests have compelling advantages. On the other
hand, if we need to test everything, we have to write
lots of unit tests. So it should be easy and pleasant
to create, manage and run them. Therefore, XP emphasizes the need for
a unit test framework that supports these tasks. In this chapter,
we will explore the use of the \code{unittest} module from
the standard Python library for creating unit tests for hardware
designs.


\section{Unit test development}

In this section, we will informally explore the application of unit
test techniques to hardware design. We will do so by a (small)
example: testing a binary to Gray encoder as introduced in
section~\ref{gray}. 

\subsection{Defining the requirements}

We start by defining the requirements. For a Gray encoder, we want to
the output to comply with Gray code characteristics. Let's define a
\dfn{code} as a list of \dfn{codewords}, where a codeword is a bit
string. A code of order \code{n} has \code{2**n} codewords.

A well-known characteristic is the one that Gray codes are all about:

\newtheorem{reqGray}{Requirement}
\begin{reqGray} 
Consecutive codewords in a Gray code should differ in a single bit.
\end{reqGray}

Is this sufficient? Not quite: suppose for example that an
implementation returns the lsb of each binary input. This would comply
with the requirement, but is obviously not what we want. Also, we don't
want the bit width of Gray codewords to exceed the bit width of the
binary codewords.

\begin{reqGray} 
Each codeword in a Gray code of order n must occur exactly once in the
binary code of the same order.
\end{reqGray}

With the requirements written down we can proceed.

\subsection{Writing the test first}

A fascinating guideline in the XP world is to write the unit test
first. That is, before implementing something, first write the test
that will verify it. This seems to go against our natural inclination,
and certainly against common practices. Many engineers like to
implement first and think about verification afterwards.

But if you think about it, it makes a lot of sense to deal with
verification first. Verification is about the requirements only --- so
your thoughts are not yet cluttered with implementation details. The
unit tests are an executable description of the requirements, so they
will be better understood and it will be very clear what needs to be
done. Consequently, the implementation should go smoother. Perhaps
most importantly, the test is available when you are done
implementing, and can be run anytime by anybody to verify changes.

Python has a standard \code{unittest} module that facilitates writing,
managing and running unit tests. With \code{unittest}, test case are
written by creating a class that inherits from
\code{unittest.TestCase}. Individual tests are created by methods of
that class: all methods that start with \code{test} are considered to
be tests of the test case.

We will define a test case for the Gray code properties, and then
write a test for each of the requirements. The outline of the test
case class is as follows:

\begin{verbatim}
from unittest import TestCase

class TestGrayCodeProperties(TestCase):

    def testSingleBitChange(self):
     """ Check that only one bit changes in successive codewords """
     ....


    def testUniqueCodeWords(self):
        """ Check that all codewords occur exactly once """
    ....

\end{verbatim}

Each method will be a small test bench that tests a single
requirement. To write the tests, we don't need an implementation of
the Gray encoder, but we do need the interface of the design. We can
specify this by a dummy implementation, as follows:

\begin{verbatim}
def bin2gray(B, G, width):
    ### NOT IMPLEMENTED YET! ###
    yield None

\end{verbatim}

For the first requirement, we will write a testbench that applies all
consecutive input numbers, and compares the current output with the
previous one for each input. Then we check that the difference is a
single bit. We will test all Gray codes up to a certain order
\code{MAX_WIDTH}.

\begin{verbatim}
    def testSingleBitChange(self):
        
        """ Check that only one bit changes in successive codewords """

        B = Signal(intbv(-1))
        G = Signal(intbv(0))
        G_Z = Signal(intbv(0))
        
        def test(width):
            B.next = intbv(0)
            yield delay(10)
            for i in range(1, 2**width):
                G_Z.next = G
                B.next = intbv(i)
                yield delay(10)
                diffcode = bin(G ^ G_Z)
                self.assertEqual(diffcode.count('1'), 1)
        
        for width in range(MAX_WIDTH):
            dut = bin2gray(B, G, width)
            sim = Simulation(dut, test(width))
            sim.run(quiet=1)

\end{verbatim}
Note how the actual check is performed by a \code{self.assertEqual}
method, defined by the \code{unittest.TestCase} class.

Similarly, we write a test bench for the second requirement. Again, we
simulate all numbers, and put the result in a list. The requirement
implies that if we sort the result list, we should get a range of
numbers:

\begin{verbatim}
    def testUniqueCodeWords(self):
        
        """ Check that all codewords occur exactly once """

        B = Signal(intbv(-1))
        G = Signal(intbv(0))

        def test(width):
            actual = []
            for i in range(2**width):
                B.next = intbv(i)
                yield delay(10)
                actual.append(int(G))
            actual.sort()
            expected = range(2**width)
            self.assertEqual(actual, expected)
       
        for width in range(MAX_WIDTH):
            dut = bin2gray(B, G, width)
            sim = Simulation(dut, test(width))
            sim.run(quiet=1)

\end{verbatim}


\subsection{Test-driven implementation}

With the test written, we begin with the implementation. For
illustration purposes, we will intentionally write some incorrect
implementations to see how the test behaves.

The easiest way to run tests defined with the \code{unittest}
framework, is to put a call to its \code{main} method at the end of
the test module:

\begin{verbatim}
unittest.main()

\end{verbatim}

Let's run the test using the dummy Gray encoder shown earlier:

\begin{verbatim}
% python test_gray.py -v
Check that only one bit changes in successive codewords ... FAIL
Check that all codewords occur exactly once ... FAIL
<trace backs not shown>

\end{verbatim}

As expected, this fails completely. Let us try an incorrect
implementation, that puts the lsb of in the input on the output:

\begin{verbatim}
def bin2gray(B, G, width):
    ### INCORRECT - DEMO PURPOSE ONLY! ###
    while 1:
        yield B
        G.next = B[0]

\end{verbatim}


Running the test produces:

\begin{verbatim}
% python test_gray.py -v
Check that only one bit changes in successive codewords ... ok
Check that all codewords occur exactly once ... FAIL

======================================================================
FAIL: Check that all codewords occur exactly once
----------------------------------------------------------------------
Traceback (most recent call last):
  File "test_gray.py", line 109, in testUniqueCodeWords
    sim.run(quiet=1)
  File "/home/jand/project/myhdl/myhdl/Simulation.py", line 87, in run
    clauses, clone = waiter.next()
  File "/home/jand/project/myhdl/myhdl/Simulation.py", line 161, in next
    clause = self.generator.next()
  File "test_gray.py", line 104, in test
    self.assertEqual(actual, expected)
  File "/usr/local/lib/python2.2/unittest.py", line 286, in failUnlessEqual
    raise self.failureException, \
AssertionError: [0, 0, 1, 1] != [0, 1, 2, 3]

----------------------------------------------------------------------
Ran 2 tests in 0.785s

\end{verbatim}

Now the test passes the first requirement, as expected, but fails the
second one. After the test feedback, a full traceback is shown that
can help to debug the test output.

Finally, if we use the correct implementation as in
section~\ref{gray}, the output is:

\begin{verbatim}
% python test_gray.py -v
Check that only one bit changes in successive codewords ... ok
Check that all codewords occur exactly once ... ok

----------------------------------------------------------------------
Ran 2 tests in 6.364s

OK

\end{verbatim}



\subsection{Changing requirements}

In the previous section, we concentrated on the general requirements
of a Gray code. It is possible to specify these without specifying the
actual code. It is easy to see that there are several codes
that satisfy these requirements. In good XP style, we only tested for
the requirements and nothing more.

It may be that more control is needed. For example, the requirement
may be for a particular code, instead of compliance with general
properties. As an illustration, we will show how to test for
\emph{the} original Gray code, which is one specific instance that
satisfies the requirements of the previous section. In this particular
case, this test will actually be easier than the previous one.

We denote the original Gray code of order \code{n} as \code{Ln}. Some
examples: 

\begin{verbatim}
L1 = ['0', '1']
L2 = ['00', '01', '11', '10']
L3 = ['000', '001', '011', '010', '110', '111', '101', 100']

\end{verbatim}

It is possible to specify these codes by a recursive algorithm, as
follows:

\begin{enumerate}
\item L1 = ['0', '1']
\item Ln+1 can be obtained from Ln as follows. Create a new code Ln0 by
prefixing all codewords of Ln with '0'. Create another new code Ln1 by
prefixing all codewords of Ln with '1', and reversing their
order. Ln+1 is the concatenation of Ln0 and Ln1.
\end{enumerate}

Python is well-known  for its elegant algorithmic
descriptions, and this is a good example. We can write the algorithm
in Python as follows:

\begin{verbatim}
def nextLn(Ln):
    """ Return Gray code Ln+1, given Ln. """
    Ln0 = ['0' + codeword for codeword in Ln]
    Ln1 = ['1' + codeword for codeword in Ln]
    Ln1.reverse()
    return Ln0 + Ln1

\end{verbatim}

The code \samp{['0' + codeword for ...]} is called a \dfn{list
comprehension}. It is a concise way to describe lists built by short
computations in a for loop.

The requirement is now that the output code matches the
expected code Ln. We use the \code{nextLn} function to compute the
expected result. The new test case code is as follows:

\begin{verbatim}
class TestOriginalGrayCode(TestCase):

    def testOriginalGrayCode(self):
        
        """ Check that the code is an original Gray code """

        B = Signal(intbv(-1))
        G = Signal(intbv(0))
        Rn = []
        
        def stimulus(n):
            for i in range(2**n):
                B.next = intbv(i)
                yield delay(10)
                Rn.append(bin(G, width=n))
        
        Ln = ['0', '1'] # n == 1
        for n in range(2, MAX_WIDTH):
            Ln = nextLn(Ln)
            del Rn[:]
            dut = bin2gray(B, G, n)
            sim = Simulation(dut, stimulus(n))
            sim.run(quiet=1)
            self.assertEqual(Ln, Rn)

\end{verbatim}

As it happens, our implementation is apparently an original Gray code:

\begin{verbatim}
% python test_gray.py -v TestOriginalGrayCode
Check that the code is an original Gray code ... ok

----------------------------------------------------------------------
Ran 1 tests in 3.091s
\end{verbatim}


 


\chapter{Reference}


\myhdl\ is implemented as a Python package called \code{myhdl}. This
chapter describes all objects that are expored by this package.

\section{The \class{Simulation} class}
\begin{classdesc}{Simulation}{*args}
Class to construct a new simulation. Each argument can either be a
\myhdl\ generator, or a nested sequence of such generators. (A nested
sequence means that each item in the sequence can itself be a
sequence.) See section~\ref{myhdl-generators} for the definition of
\myhdl\ generators.
\end{classdesc}

A \class{Simulation} instance has the following method:

\begin{methoddesc}[Simulation]{run}{\optional{duration}}
Run the simulation forever or for a specified duration.
\end{methoddesc}

\section{The \class{Signal} class}
\begin{classdesc}{Signal}{val, \optional{delay}}

A \class{Signal} has the following attributes:

\begin{memberdesc}[Signal]{val}
Read-only attribute that represents the current value of the signal. 

\end{memberdesc}
\begin{memberdesc}[Signal]{next}
Read-write attribute.

\end{memberdesc}

\end{classdesc}

\section{\myhdl\ generators}
\label{myhdl-generators}


\section{Modeling convenience objects}

\begin{excclassdesc}{StopStimulation}{}
Base exception that is caught by the \code{Simulation.run} method to
stop a simulation. Can be subclassed and raised in generator code.
\end{excclassdesc}

\begin{funcdesc}{now}{}
Return the current simulation time.
\end{funcdesc}

\begin{funcdesc}{downrange}{high, \optional{low}}
Generates a downward range list. Like the standard \code{range}
function, but in the downward direction.
\end{funcdesc}

\section{The  \class{intbv} class}








\end{document}
