\chapter{Reference}


\myhdl\ is implemented as a Python package called \code{myhdl}. This
chapter describes all objects that are expored by this package.

\section{The \class{Simulation} class}
\begin{classdesc}{Simulation}{*args}
Class to construct a new simulation. Each argument can either be a
\myhdl\ generator, or a nested sequence of such generators. (A nested
sequence means that each item in the sequence can itself be a
sequence.) See section~\ref{myhdl-generators} for the definition of
\myhdl\ generators.
\end{classdesc}

A \class{Simulation} instance has the following method:

\begin{methoddesc}[Simulation]{run}{\optional{duration}}
Run the simulation forever or for a specified duration.
\end{methoddesc}

\section{The \class{Signal} class}
\begin{classdesc}{Signal}{val, \optional{delay}}

A \class{Signal} has the following attributes:

\begin{memberdesc}[Signal]{val}
Read-only attribute that represents the current value of the signal. 

\end{memberdesc}
\begin{memberdesc}[Signal]{next}
Read-write attribute.

\end{memberdesc}

\end{classdesc}

\section{\myhdl\ generators}
\label{myhdl-generators}


\section{Modeling convenience objects}

\begin{excclassdesc}{StopStimulation}{}
Base exception that is caught by the \code{Simulation.run} method to
stop a simulation. Can be subclassed and raised in generator code.
\end{excclassdesc}

\begin{funcdesc}{now}{}
Return the current simulation time.
\end{funcdesc}

\begin{funcdesc}{downrange}{high, \optional{low}}
Generates a downward range list. Like the standard \code{range}
function, but in the downward direction.
\end{funcdesc}

\section{The  \class{intbv} class}






